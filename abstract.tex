% Complete abstract must be < 3500 characters, so may need shortening.
% I will also submit an extended abstract, so a longer version is fine to draft. I can cut that down.

Computer software has become an indispensable resource in modern neuroscience. From the gathering of data, simulation of computational models, analysis of large amounts of information, collaboration, and communication tools for community development, software is now a necessary part of the research pipeline.

The software tools used in Neuroscience and research are generally sophisticated and involve many pre-processing steps limiting the portability of the software. In the versatile community of Neuroscience, many researchers are not well acquainted with computing and find it challenging to install, use, and manage such software tools.

The Neuroscience community is gradually moving to the use of Free/Open Source software (FOSS)~\cite{Gleeson2017} and therefore, we present a ready to use, FOSS platform for Neuroscientists.
We have leveraged the infrastructure resources of the FOSS Fedora community~\cite{RedHat2008} to develop an operating system that includes a plethora of ready-to-use Neuroscience software.
All software included in NeuroFedora is built following the best software development practices and follows the Fedora community's Quality Assurance process. Furthermore, NeuroFedora is well integrated with other software such as desktop environments, text editors, and other daily use and development tools.

While work continues to make more software available in NeuroFedora covering all aspects of Neuroscience---including computational modelling, data analysis, and neuro-imaging---NeuroFedora already provides commonly used Computational Neuroscience tools, such as the NEST simulator~\cite{Linssen2018}, GENESIS~\cite{Bower2003}, Auryn~\cite{Zenke2014}, Neuron~\cite{Hines1997}, Brian (v1 and v2)~\cite{Goodman2009}, Moose~\cite{Dudani2009}, Neurord~\cite{Jedrzejewski2016}, Bionetgen~\cite{Harris2016}, COPASI~\cite{Mendes2009}, PyLEMS~\cite{Vella2014}, and others.

With up to date documentation at \url{neuro.fedoraproject.org}, we invite researchers to use NeuroFedora in their research and to join the team to help NeuroFedora better aid the research community.