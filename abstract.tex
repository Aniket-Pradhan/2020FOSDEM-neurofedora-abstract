% Complete abstract must be < 3500 characters, so may need shortening.
% I will also submit an extended abstract, so a longer version is fine to draft. I can cut that down.

Modern Neuroscience relies heavily on software.
From the gathering of data, the simulation of computational models, the analysis of large amounts of information, to collaboration and communication tools for community development, software is now a necessary part of the Neuroscience research pipeline.

While the Neuroscience community is increasingly moving to the use of Free/Open Source software~\cite{Stallman2002,Gleeson2017}, a majority of the tools are complex and not trivial to deploy.
In a community that is as multidisciplinary as Neuroscience, a large chunk of researchers hail from fields other than computing.
It, therefore, often requires considerable time and effort to install, configure, and maintain tools used in research.

In NeuroFedora, we present a ready to use, Free/Open Source platform for Neuroscientists to use.
We leverage the infrastructure resources of the Free/Open Source Fedora community~\cite{RedHat2008}\commentAnkursinha{If we get a new citation from Mindshare, we'll use that} to develop a ready to install Linux distribution that includes a plethora of Neuroscience software.
All software included in NeuroFedora is built in accordance with modern software development best practices, follows the Fedora community's Quality Assurance (QA) process, and is well integrated with other software such as desktop environments, text editors, and other daily use and development tools.

While NeuroFedora is only a few months old, it already provides commonly used tools in Computational Neuroscience, such as the NEST simulator~\cite{Linssen2018}, GENESIS~\cite{Bower2003}, Auryn~\cite{Zenke2014}, Neuron~\cite{Hines1997}, Brian (versions 1 and 2)~\cite{Goodman2009}, Moose~\cite{Dudani2009}, Neurord, Bionetgen~\cite{Harris2016}, COPASI~\cite{Mendes2009}, PyLEMS~\cite{Vella2014}, and many others.
Where supported, variants of these tools that support super computers and clusters using the Message Passing Interface (MPI) are also supplied.

Work continues to make more software available in NeuroFedora, including software used in data analysis and Neuro-imaging.
With up to date documentation at https://neuro.fedoraproject.org, we invite researchers to use NeuroFedora in their research and to join the team to help NeuroFedora better aid the research community.
